
% this file is called up by thesis.tex
% content in this file will be fed into the main document

\chapter{Conclusions}\label{cap:conc} % top level followed by section, subsection


% ----------------------- paths to graphics ------------------------

% change according to folder and file names
\ifpdf
    \graphicspath{{8_conclusions/figures/PNG/}{8_conclusions/figures/PDF/}{8_conclusions/figures/}}
\else
    \graphicspath{{8_conclusions/figures/EPS/}{8_conclusions/figures/}}
\fi

% ----------------------- contents from here ------------------------

Amb aquest projecte hem pogut aproximar-nos als Sistemes Distribuïts de Control, i veure els problemes que aquests sistemes arrosseguen. Al principi podia ser una mica desconcertant, però un cop ens hi endinsem les coses comencen a agafar un cert sentit i es comencen a veure d'una altre manera.

En el projecte finalment hem pogut complir tots els requisits plantejats des de bon principi.

Hem pogut agafar i entendre tot el codi del laboratori actual, per d'aquesta manera crear tot l'entorn de codi lliure.

Hem dissenyat l'entorn de laboratori d'un Sistema Distribuït de Control real, en el qual tots els dispositius comparteixen realment el mateix bus CAN, i d'aquesta manera els alumnes poden aprendre en un entorn conflictiu real. 

Per dissenyar aquest laboratori hem hagut de organitzar els diferents identificadors que hi ha permesos al protocol CAN, de manera que existeixi una jerarquia de prioritats entre els missatges CAN, que una part pugui identificar l'identificador del llaç de control, i una altra quin tipus de missatge és. Tot això s'ha dissenyat seguint aquests requisits, i finalment els identificadors tenen una estructura totalment ben definida.

Aquest nou laboratori ha comportat la creació d'un nou programa per ordinador; el qual hem anomenat \DCSMonitor; i que hem dissenyat per complir tots els objectius que es van plantejar. 

D'aquesta manera hem estudiat i creat tot el codi necessari perquè aquest programa fos capaç de comunicar-se a traves del port serie via RS232 a una placa \FLEX per tal de: demanar un llistat de tots els llaços de control existents al bus CAN, demanar-li que carregués el bus de manera que els llaços de control es veiessin amb problemes reals, demanar-li totes les dades de l'estat d'un llaç de control i d'aquesta manera poder avaluar la bondat del control de cada grup del laboratori. I tot això mantenint la compatibilitat del laboratori antic, creant dos modes d'execució, un en el que es totalment compatible amb les plaques \FLEX de l'antic laboratori i on cada alumne pot avaluar el seu control, i un mode d'execució en el que intervé un dispositiu nou que anomenem \Monitor, amb el que es poden fer totes les accions anteriorment comentades.

Una peculiaritat de la comunicació serie que s'ha dissenyat es que s'ha realitzat de manera que pugui ser fàcilment ampliable i reutilitzable per tal de afegir més opcions per al \Monitor o per el programa \DCSMonitor.

Amb les dades rebudes d'un llaç de control connectat al programa, o monitoritzat per un dispositiu \Monitor, hem fet que el programa \DCSMonitor sigui capaç de generar unes gràfiques en temps real amb els valors del control de: referència, valor d'entrada del doble integrador i valors de la primera i segona integral. A més aquests plots poden ser trets de la gràfica individualment.

Les gràfiques en temps real també poden ser exportades a diferents tipus d'imatge, entre d'elles imatges vectorials, molt interessants per afegir en documents o articles.

Un atre dels requisits del projecte era que el programa fos multilingüe, i ho hem complert creant l'entorn disponible en català, castellà, anglès i francès. A més hem utilitzat una metodologia de programació modular que ens permet fàcilment afegir tots els idiomes que vulguem. 

A part del programa d'ordinador \DCSMonitor, també calia crear un dispositiu que pogués monitoritzar els llaços de control que es trobessin al bus CAN (dispositiu que hem anomenat \Monitor). Per tal s'ha dissenyat el sistema de recepció de missatges per RS232 del programa \DCSMonitor, activant interrupcions i comprovant quin tipus de acció es volia portar a terme, i posant en marxa diferents tasques dependents d'això.

Els dispositius que ja existien en el laboratori actual s'han mantingut totalment compatibles, i s'els ha afegit el necessari per poder compartir el bus CAN sense conflictes, creant l'estructura d'identificadors de manera modular, i havent de incloure únicament el numero de grup al que pertanyen.

Sobre les guies, s'ha preparat una guia de laboratori per posar l'entorn de laboratori necessari per poder executar les diferents práctiques que des de la universitat s'imparteixen.

També s'ha creat un \LiveCD amb un Linux \Ubuntu, amb tot el codi necessari preparat per ser arrencat i provat (el pes de tot el sistema amb \Eclipse i \MplabX inclosos han fet que el pes total del sistema sigui de 1,4GB el que ha comportat que el \LiveCD s'hagi de gravar en un DVD o en un llapis de memòria). D'aquesta manera només logejar-nos podem programar el llaç de control, i començar a fer les lectures. I la guia d'aquest CD que ens explica pas a pas com poder fer una instal·lació de l'entorn en un ordinador o en una màquina virtual, i després com utilitzar-lo tant en l'entorn instal·lat com des de el \LiveCD.

Per tant un cop finalitzat han estat complerts tots els requisits proposats en un principi, quedant constància de manera desglossada al llarg de tota la memòria. 


Cal destacar que l'entrega d'aquest projecte no tanca aquest laboratori, ja que el disseny que s'ha realitzat durant el projecte ha estat encaminat cap al doble integrador, però s'ha programat l'entorn, i s'han pensat tots els identificadors i missatges de manera que pugui ser ampliat per atendre altre tipus de controls, realitzar diferents accions des de l'ordinador, deixant espai suficient perquè la comunicació entre el programa \DCSMonitor i el dispositiu \Monitor tinguin una gran varietat de senyals diferents i es pugui modificar el laboratori de varies maneres.
 
\subsection{Treball futur}

En el transcurs del projecte han anat apareixen diverses funcions que podien ampliar el projecte però quedaven fora de l'abast d'aquest, així que posteriorment a l'entrega d'aquest projecte encara es poden realitzar aquestes ampliacions i/o millores:

\begin{enumerate}
	\item Ampliar el nombre d'idiomes en el que està traduït el programa per l'ordinador.
	\item Traduir les guies a altres idiomes.
	\item Que el programa \DCSMonitor sigui capaç de mostrar les gràfiques de més d'un llaç de control al mateix temps.
	\item Que el dispositiu \Monitor del microcontrolador sigui capaç d'estimar el percentatge de saturació del bus CAN, i d'aquesta manera poder enviar aquesta informació al programa \DCSMonitor.
	\item Crear el codi necessari perquè els missatges CAN no provoquin interrupcions, sinó que en tot moment es faci un pooling controlat.
	\item Mirar de programar el microcontrolador amb una eina més senzilla i menys pesada que el \MplabX, el qual actualment només s'utilitza per aquest fi.
	\item Reduir la mida de la imatge del laboratori perquè càpiga en un CD.
	\item Utilitzar les tres màscares CAN disponibles en els microcontroladors.
	\item En el programa \DCSMonitor separar en diferents procesos l'adquisició de dades del port serie i el dibuix de la gràfica.
	\item Organitzar de manera eficient els textos per traduir del programa \DCSMonitor
\end{enumerate}


% ---------------------------------------------------------------------------
%: ----------------------- end of thesis sub-document ------------------------
% ---------------------------------------------------------------------------



 






